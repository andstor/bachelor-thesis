%This is chapter 1
%%=========================================
\chapter[Introductions]{Introduction}

This report will review different ...
...\\

%%=========================================
\section{Background}
Since the introduction of the relatively new technology WebGL, users has been able to significantly expand the graphical user-experience for the large user mass of the web. It has allowed for almost desktop like / native graphics performance, 

WebGl has especially expanded the abilities regarding the creation of web browser games and simulations. Creating high performance graphics applications in the browser has never been easier. It has opened up for a sea of possibilities where only your imagination is the limit.

In a lot of simulation software, and even some games, volumetric information plays a crucial part. With everything from fluid dynamics to voxel games like Minecraft, volumetric information is a key component. However, to the best of my knowledge, it does not exist any easy to use open-source voxelization software in JavaScript. In order to obtain such volumetric data, developers are therefore forced to go through a tedious preprocessing steps, often involving old and complex, har to use, platform specific tools.

This was a problem I encountered myself a year ago in 2019. In connection with an assignment in a simulation course at the Norwegian University of Science and Technology, I needed to be able to easily generate some volumetric data based on 3D models. I was using web-technologies so I was looking for a simple solution in plain JavaScript. However, I was not able to find such a solution. I therefore decided to make one myself. The result was an open source voxelization project, written entirely in JavaScript, named Voxelizer.

The Voxelizer project carries strong signs of the limited amount of time allocated for sowing the software together. PROBLEMS HERE........ Due to it's vast range of applicability, improving and expanding the capabilities of the project could therefore serve to be a valuable open-source asset to the web based game- and simulation-development ecosystem, providing easy access to voxelization.

%==========================================
\section{Problem Formulation}

\subsubsection{Problems to be addressed}
There exists an open-source JavaScript voxelization engine for voxelizing 3D models. The project faces several issues and is lacking important features. The engine does not produce accurate and representative results. The output sometimes contains holes and contains a lot of artifacts. Importing support is limited to OBJ files. In terms of exporting, only 3D arrays are supported. [SHORTEN THIS!!!!].. Documentation is lacking, and the coding quality is relatively poor. The project needs to be professionalized, and made easy to both use and maintain.

Something about automation, maintainability?

\begin{itemize}
\item Design ..
\item Develop ...
\end{itemize}

%%=========================================
\section{Literature Survey}
This report is based on ....  published in 2016.
%%=========================================
\section{Objectives}
The main goals of this thesis is to improve and extend the open-source JavaScript Voxelizer engine, turning it into a maintainable, high-quality, open-source project. A secondary goal will be to develop complimentary software for the Voxelizer project. This will be in the form of a cross platform desktop application and CLI, based on the Voxelizer engine, making it easy to voxelize 3D models.

In order to ensure the maintainability of the various software projects, automation is a critical component. Therefore a common subgoal will be to develop a GitHub Action for automating the API documentation generation process.

SOMETHING ABOUT REQUIREMENT SPECIFICATION????
Thus, the objectives for this report thesis are: ????????
\begin{enumerate}
\item Make ...
\item Implement ...
\item Create ...
\end{enumerate}

%%=========================================
\section{Scope}
In this particular project .....  

The main purpose of this thesis is to make it easy to conduct high quality voxelization of 3D models. 

It is important to note that the project does not mainly focus on speed of the voxelization algorithm. The targeted systems often runs in an environment where resources are scarce and performance is important. However, usability is also extremely important. This thesis will mainly focus on providing easy access to high-quality voxelization, within a reasonable performance. If speed is of the main concerns, it would be better to do the extra work i setting up native solutions, often written in C/C++.



%%=========================================
\section{Structure}
\subsection{Voxelization systems overview}
The diagram in figure \ref{fig:systems-overview} shows how the different software repositories regarding voxelization interconnects. The green boxes represents the main software projects. The blue box represents side projects. During development, some components were generalized and extracted into a separate repository.
%\clearpage % remove this ??
\begin{figure}[h]
    \centering
    \includegraphics[page=1,scale=1]{sections/introduction/figures/systems-overview.pdf}
    \caption{Automation of release publishing process.}
    \label{fig:systems-overview}
\end{figure}

\subsection{Automation overview}
\colorbox{RubineRed}{Include diagram of simple github automation prosess?}

%%=========================================
\section{Outline}

The rest of the report is structured as follows.\\
\break
\textbf{Chapter 2 - Theory:} Chapter two gives an introduction to the theoretical background that lies the foundation of this thesis.\\
\break
\textbf{Chapter 3 - Method:} Contains a description of the methodology and materials that were considered throughout the project.\\
\break
\textbf{Chapter 4 - Result:} Contains a description of the finished software systems.\\
\break
\textbf{Chapter 5 - Discussion:} Discusses the achieved results, the execution of methodologies and tools, in addition to encountered difficulties.\\
\break
\textbf{Chapter 6 - Conclusions:} This chapter presents an overall conclusion of the project, reviewing the objectives and the progress made.\\
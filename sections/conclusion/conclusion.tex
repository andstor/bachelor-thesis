%This is chapter 6
%%=========================================
\chapter{Conclusion}
The primary objective of this thesis was to improve the Voxelizer engine. The engine is completely overhauled, addressing all known issues with the old version. Several new features are also added, like coloring and shell voxelization. The performance of the engine is also greatly improved. The speed of the voxelization is significantly increased, and the memory footprint is reduced by several orders of magnitudes. The exporting capabilities are also extended with several file formats and data structures. This includes ndarrays, XML, and BINVOX. Exporting to BINVOX files is made possible with the new BINVOX JavaScript package.

To make voxelization easy and accessible, a complementary cross platform desktop application is developed for the Voxelizer engine. This features a sleek drag and drop interface, along with visualized voxelization results, made possible with the new three-voxel-loader three.js plugin.

To ensure the projects keep a high level of quality, and are easy to maintain, a lot of automation is implemented. This has removed much manual and laborious work. It has also made the software a lot safer, drastically reducing the potential of human errors and bug introductions. To achieve this level of automation, several GitHub Actions are developed. This mainly includes the highly successful documentation tool, the JSDoc Action. It is embraced by the popular JSDoc tool, and is already gaining in popularity. Two more actions are also developed for automation purposes. This is the File Existence action, and the File Reader action.   

To summarize, the Voxelizer engine is greatly improved. A companion desktop application is developed, alongside several automation tools. The project is considered a success, and I believe that these results are a useful contribution to the open-source community.